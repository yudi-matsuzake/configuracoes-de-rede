\PassOptionsToPackage{unicode=true}{hyperref} % options for packages loaded elsewhere
\PassOptionsToPackage{hyphens}{url}
%
\documentclass[]{article}
\usepackage{lmodern}
\usepackage{amssymb,amsmath}
\usepackage{ifxetex,ifluatex}
\usepackage{fixltx2e} % provides \textsubscript
\ifnum 0\ifxetex 1\fi\ifluatex 1\fi=0 % if pdftex
  \usepackage[T1]{fontenc}
  \usepackage[utf8]{inputenc}
  \usepackage{textcomp} % provides euro and other symbols
\else % if luatex or xelatex
  \usepackage{unicode-math}
  \defaultfontfeatures{Ligatures=TeX,Scale=MatchLowercase}
\fi
% use upquote if available, for straight quotes in verbatim environments
\IfFileExists{upquote.sty}{\usepackage{upquote}}{}
% use microtype if available
\IfFileExists{microtype.sty}{%
\usepackage[]{microtype}
\UseMicrotypeSet[protrusion]{basicmath} % disable protrusion for tt fonts
}{}
\IfFileExists{parskip.sty}{%
\usepackage{parskip}
}{% else
\setlength{\parindent}{0pt}
\setlength{\parskip}{6pt plus 2pt minus 1pt}
}
\usepackage{hyperref}
\hypersetup{
            pdfborder={0 0 0},
            breaklinks=true}
\urlstyle{same}  % don't use monospace font for urls
\usepackage{graphicx,grffile}
\makeatletter
\def\maxwidth{\ifdim\Gin@nat@width>\linewidth\linewidth\else\Gin@nat@width\fi}
\def\maxheight{\ifdim\Gin@nat@height>\textheight\textheight\else\Gin@nat@height\fi}
\makeatother
% Scale images if necessary, so that they will not overflow the page
% margins by default, and it is still possible to overwrite the defaults
% using explicit options in \includegraphics[width, height, ...]{}
\setkeys{Gin}{width=\maxwidth,height=\maxheight,keepaspectratio}
\setlength{\emergencystretch}{3em}  % prevent overfull lines
\providecommand{\tightlist}{%
  \setlength{\itemsep}{0pt}\setlength{\parskip}{0pt}}
\setcounter{secnumdepth}{0}
% Redefines (sub)paragraphs to behave more like sections
\ifx\paragraph\undefined\else
\let\oldparagraph\paragraph
\renewcommand{\paragraph}[1]{\oldparagraph{#1}\mbox{}}
\fi
\ifx\subparagraph\undefined\else
\let\oldsubparagraph\subparagraph
\renewcommand{\subparagraph}[1]{\oldsubparagraph{#1}\mbox{}}
\fi

% set default figure placement to htbp
\makeatletter
\def\fps@figure{htbp}
\makeatother


\date{}

\begin{document}

{
\setcounter{tocdepth}{3}
\tableofcontents
}
\hypertarget{acl}{%
\section{ACL}\label{acl}}

\hypertarget{processo-resumido-para-aplicauxe7uxe3o-da-acl}{%
\subsection{Processo resumido para aplicação da
ACL}\label{processo-resumido-para-aplicauxe7uxe3o-da-acl}}

\begin{enumerate}
\def\labelenumi{\arabic{enumi}.}
\item
  Criar a restrição;
\item
  Aplicar a restrição a interface;
\item
  Informar a direção:

  \begin{itemize}
  \tightlist
  \item
    inbound (entrando);
  \item
    outbound (saindo).
  \end{itemize}
\end{enumerate}

\hypertarget{criando-a-access-list}{%
\subsection{Criando a Access-List}\label{criando-a-access-list}}

\hypertarget{padruxe3o}{%
\subsubsection{Padrão}\label{padruxe3o}}

\begin{itemize}
\item
  Se você colocar \texttt{host} antes do ip, ele vai diretamente pra
  máquina;
\item
  Se não, deve colocar o ip da rede e uma wild mask;
\end{itemize}

\begin{verbatim}
R (config)# access list 1 {permit|deny} host xxx.xxx.xxx.xxx
\end{verbatim}

\hypertarget{extendida}{%
\subsubsection{Extendida}\label{extendida}}

\begin{verbatim}
R (config)# access-list 100 {permit|deny} ip [ {host|protocolo} xxx.xxx.xxx.xxx ] \
        [ xxx.xxx.xxx.xxx wildmask]
\end{verbatim}

\hypertarget{adicionando-a-acl-em-uma-interface}{%
\subsection{Adicionando a ACL em uma
interface}\label{adicionando-a-acl-em-uma-interface}}

\begin{verbatim}
R> interface serial 0/0/0
R (config)# ip access-group 1 {in|out}
\end{verbatim}

\#DHCP \# Etherchannels

\begin{itemize}
\tightlist
\item
  O administrador deve indicar uma interface para o etherchanel usando o
  comando channel-group;
\end{itemize}

\hypertarget{pagp-vs-lacp}{%
\subsection{PAgP vs LACP}\label{pagp-vs-lacp}}

\begin{itemize}
\item
  PAgP é da cisco
\item
  LACP IEEE 802.3ad standard
\end{itemize}

\begin{figure}
\centering
\includegraphics{imagens/etherchannel/PAgPvsLAcP.png}
\caption{PAgP vs LACP}
\end{figure}

\hypertarget{comandos}{%
\subsection{Comandos}\label{comandos}}

\begin{itemize}
\item
  Channel group
\item
  Show etherchannel
\item
  Show pagp
\end{itemize}

\includegraphics{imagens/etherchannel/etherchannel1.png}
\includegraphics{imagens/etherchannel/etherchannel2.png}
\includegraphics{imagens/etherchannel/etherchannel3.png}

\hypertarget{configurauxe7uxe3o}{%
\subsection{Configuração}\label{configurauxe7uxe3o}}

\hypertarget{auxe7uxf5es}{%
\subsubsection{Ações}\label{auxe7uxf5es}}

\begin{figure}
\centering
\includegraphics{imagens/etherchannel/acoes.png}
\caption{Ações}
\end{figure}

\textbf{Comandos}

\begin{verbatim}
Switch(config)# interface range interface slot/port - port
Switch(config-if-range)# channel-protocol {pagp | lacp}
Switch(config-if-range)# channel-group number mode {auto|disirable|on}
\end{verbatim}

\textbf{Modos}

\begin{figure}
\centering
\includegraphics{imagens/etherchannel/modos.png}
\caption{Modos}
\end{figure}

\textbf{Estático}

\begin{figure}
\centering
\includegraphics{imagens/etherchannel/estatico.png}
\caption{Estático}
\end{figure}

\textbf{PAgP}

\begin{figure}
\centering
\includegraphics{imagens/etherchannel/PAgP.png}
\caption{PAgP}
\end{figure}

\textbf{LACP}

\includegraphics{imagens/etherchannel/LACP.png} \# PVC frame relay

\hypertarget{configurauxe7uxe3o-buxe1sica-de-um-pvc-frame-relay-pvc-em-um-roteador-com-interface-serial.}{%
\subsection{Configuração básica de um PVC Frame Relay PVC em um roteador
com interface
serial.}\label{configurauxe7uxe3o-buxe1sica-de-um-pvc-frame-relay-pvc-em-um-roteador-com-interface-serial.}}

\begin{figure}
\centering
\includegraphics{imagens/frame-relay/pvc-basico-frame-relay.png}
\caption{Configuração Básica}
\end{figure}

\hypertarget{configurauxe7uxe3o-de-um-mapa-estuxe1tico-frame-relay}{%
\subsection{Configuração de um mapa estático Frame
Relay}\label{configurauxe7uxe3o-de-um-mapa-estuxe1tico-frame-relay}}

\begin{figure}
\centering
\includegraphics{imagens/frame-relay/pvc-basico-frame-relay0.png}
\caption{Configuração Básica}
\end{figure}

\hypertarget{configurauxe7uxe3o-de-uma-topologia-de-rede-frame-relay-hub-and-spoke}{%
\section{Configuração de uma topologia de rede Frame Relay Hub and
Spoke}\label{configurauxe7uxe3o-de-uma-topologia-de-rede-frame-relay-hub-and-spoke}}

\begin{figure}
\centering
\includegraphics{imagens/frame-relay/frame-raley-hub-and-spoke.png}
\caption{Configuração Hub and Spoke}
\end{figure}

\hypertarget{configurauxe7uxe3o-do-roteador-da-matriz}{%
\subsection{Configuração do Roteador da
Matriz}\label{configurauxe7uxe3o-do-roteador-da-matriz}}

\begin{verbatim}
Router#config terminal
Router(config)#hostname Matriz
Matriz(config)#interface serial 0/0
Matriz(config-if)#encapsulation frame-relay
Matriz(config-if)#no frame-relay inverse-arp
Matriz(config-if)#no shut
Matriz(config-if)#exit
Matriz(config)#interface serial 0/0.102 point-to-point
Matriz(config-subif)#ip address 200.1.1.1 255.255.255.252
Matriz(config-subif)#frame-relay interface-dlci 102
Matriz(config-fr-dlci)#exit
Matriz(config-subif)#exit
Matriz(config)#interface serial 0/0.103 point-to-point
Matriz(config-subif)#ip address 200.1.1.5 255.255.255.252
Matriz(config-subif)#frame-relay interface-dlci 103
Matriz(config-fr-dlci)#exit
Matriz(config-subif)#exit
Matriz(config)#interface loopback 0
Matriz(config-if)#ip address 200.10.1.1 255.255.255.0
Matriz(config-if)#no shut
Matriz(config-if)#exit
Matriz(config)#
\end{verbatim}

\hypertarget{configurauxe7uxe3o-do-roteador-da-filial-1}{%
\subsection{Configuração do roteador da Filial
1}\label{configurauxe7uxe3o-do-roteador-da-filial-1}}

\begin{verbatim}
Router>enable
Router#config terminal
Router(config)#hostname Filial1
Filial1(config)#interface serial 0/2
Filial1(config-if)#encapsulation frame-relay
Filial1(config-if)#no frame-relay inverse-arp
Filial1(config-if)#no shutdown
Filial1(config-if)#exit
Filial1(config)#
Filial1(config)#interface serial 0/2.201 point-to-point
Filial1(config-subif)#
Filial1(config-subif)#ip address 200.1.1.2 255.255.255.252
Filial1(config-subif)#frame-relay interface-dlci 201
Filial1(config-fr-dlci)#exit
Filial1(config-subif)#exit
Filial1(config)#interface loopback 0
Filial1(config-if)#ip address 200.134.10.1 255.255.255.0
Filial1(config-if)#end
Filial1#
\end{verbatim}

\hypertarget{configurauxe7uxe3o-do-roteador-da-filial-2}{%
\subsection{Configuração do roteador da Filial
2}\label{configurauxe7uxe3o-do-roteador-da-filial-2}}

\begin{verbatim}
Router>enable
Router#config terminal
Router(config)#hostname Filial2
Filial2(config)#interface serial 0/0
Filial2(config-if)#encapsulation frame-relay
Filial2(config-if)#no frame-relay inverse-arp
Filial2(config-if)#no shut
Filial2(config-if)#clock rate 640
Filial2(config-if)#exit
Filial2(config)#interface serial 0
Filial2(config)#interface serial 0/0.301 point-to-point
Filial2(config-subif)#ip address 200.1.1.6 255.255.255.252
Filial2(config-subif)#frame-relay interface-dlci 301
Filial2(config-fr-dlci)#exit
Filial2(config-subif)#exit
Filial2(config)#interface loopback 0
Filial2(config-if)#ip address 200.100.1.1 2
Filial2(config-if)#ip address 200.100.1.1 255.255.255.0
Filial2(config-if)#no shut
Filial2(config-if)#end
Filial2#
\end{verbatim}

\hypertarget{configurauxe7uxe3o-do-switch-frame-relay-1}{%
\subsection{Configuração do Switch Frame Relay
1}\label{configurauxe7uxe3o-do-switch-frame-relay-1}}

\begin{verbatim}
SFR1#config terminal
SFR1(config)#hostname SFR1
SFR1(config)#frame-relay switching
SFR1(config)#interface serial 0/0
SFR1(config-if)#encapsulation frame-relay
SFR1(config-if)#no frame-relay inverse-arp
SFR1(config-if)#frame-relay intf-type dce
SFR1(config-if)#frame-relay route 102 interface serial 0/1 202
SFR1(config-if)#frame-relay route 103 interface serial 0/3 303
SFR1(config-if)#clock rate 64000
SFR1(config-if)#no shut
SFR1(config-if)#exit
SFR1(config)#interface serial 0/1
SFR1(config-if)#encapsulation frame-relay
SFR1(config-if)#no frame-relay inverse-arp
SFR1(config-if)#frame-relay intf-type nni
SFR1(config-if)#frame-relay route 202 interface serial 0/0 102
SFR1(config-if)#clock rate 64000
SFR1(config-if)#no shut
SFR1(config-if)#exit
SFR1(config)#interface serial 0/3
SFR1(config-if)#encapsulation frame-relay
SFR1(config-if)#no frame-relay inverse-arp
SFR1(config-if)#frame-relay intf-type nni
SFR1(config-if)#frame-relay route 303 interface serial 0/0 103
SFR1(config-if)#clock rate 64000
SFR1(config-if)#no shut
SFR1(config-if)#end
SFR1#
\end{verbatim}

\hypertarget{configurauxe7uxe3o-do-switch-frame-relay-2}{%
\subsection{Configuração do Switch Frame Relay
2}\label{configurauxe7uxe3o-do-switch-frame-relay-2}}

\begin{verbatim}
SFR2#config terminal
SFR2(config)#frame-relay switching
SFR2(config)#interface serial 0/1
SFR2(config-if)#encapsulation frame-relay
SFR2(config-if)#no frame-relay inverse-arp
SFR2(config-if)#clock rate 64000
SFR2(config-if)#frame-relay intf-type nni
SFR2(config-if)#frame-relay route 202 interface serial 0/2 201
SFR2(config-if)#no shut
SFR2(config-if)#exit
SFR2(config)#
SFR2(config)#interface serial 0/2
SFR2(config-if)#encapsulation frame-relay
SFR2(config-if)#no frame-relay inverse-arp
SFR2(config-if)#frame-relay intf-type dce
SFR2(config-if)#frame-relay route 201 interface serial 0/1 202
SFR2(config-if)#clock rate 64000
SFR2(config-if)#no shut
SFR2(config-if)#
SFR2(config-if)#end
SFR2#wr
\end{verbatim}

\hypertarget{configurauxe7uxe3o-do-switch-frame-relay-3}{%
\subsection{Configuração do Switch Frame Relay
3}\label{configurauxe7uxe3o-do-switch-frame-relay-3}}

\begin{verbatim}
SFR3#configure terminal
SFR3(config)#frame-relay switching
SFR3(config)#interface serial 0/3
SFR3(config-if)#encapsulation frame-relay
SFR3(config-if)#no frame-relay inverse-arp
SFR3(config-if)#clock rate 64000
SFR3(config-if)#frame-relay intf-type nni
SFR3(config-if)#frame-relay route 303 interface serial 0/0 301
SFR3(config-if)#no shut
SFR3(config-if)#exit
SFR3(config)#interface serial 0/0
SFR3(config-if)#encapsulation frame-relay
SFR3(config-if)#no frame-relay inverse-arp
SFR3(config-if)#frame-relay intf-type dce
SFR3(config-if)#clock rate 64000
SFR3(config-if)#frame-relay route 301 interface serial 0/3 303
SFR3(config-if)#no shutConfiguração do Roteador da Matriz
Router#config terminal
Router(config)#hostname Matriz
Matriz(config)#interface serial 0/0
Matriz(config-if)#encapsulation frame-relay
Matriz(config-if)#no frame-relay inverse-arp
Matriz(config-if)#no shut
Matriz(config-if)#exit
Matriz(config)#interface serial 0/0.102 point-to-point
Matriz(config-subif)#ip address 200.1.1.1 255.255.255.252
Matriz(config-subif)#frame-relay interface-dlci 102
Matriz(config-fr-dlci)#exit
Matriz(config-subif)#exit
SFR3(config-if)#end
SFR3#
\end{verbatim}

\hypertarget{configurauxe7uxe3o-do-protocolo-de-roteamento-ospf}{%
\subsection{Configuração do protocolo de roteamento
OSPF}\label{configurauxe7uxe3o-do-protocolo-de-roteamento-ospf}}

\begin{verbatim}
Matriz
Matriz#config terminal
Matriz(config)#router ospf 1
Matriz(config-router)#network 200.10.1.0 0.0.0.255 area 0
Matriz(config-router)#network 200.1.1.0 0.0.0.3 area 0
Matriz(config-router)#network 200.1.1.4 0.0.0.3 area 0
Matriz(config-router)#end
\end{verbatim}

\hypertarget{filial-1}{%
\subsection{Filial 1}\label{filial-1}}

\begin{verbatim}
Filial1>enable
Filial1#config terminal
Filial1(config)#route ospf 1
Filial1(config)#router ospf 1
Filial1(config-router)#network 200.134.10.0 0.0.0.255 area 0
Filial1(config-router)#network 200.1.1.0 0.0.0.3 area 0
Filial1(config-router)#end
Filial1#
\end{verbatim}

\hypertarget{filial-2}{%
\subsection{Filial 2}\label{filial-2}}

\begin{verbatim}
Filial2>enable
Filial2#config terminal
Filial2(config)#router ospf 1
Filial2(config-router)#network 200.100.1.0 0.0.0.255
Filial2(config-router)#network 200.100.1.0 0.0.0.255 area 0
Filial2(config-router)#network 200.1.1.4 0.0.0.3 area 0
Filial2(config-router)#end
Filial2#
\end{verbatim}

\hypertarget{verificando-as-configurauxe7uxf5es}{%
\subsection{Verificando as
configurações:}\label{verificando-as-configurauxe7uxf5es}}

\begin{verbatim}
Matriz#show frame-relay map
Serial0/0.102 (up): point-to-point dlci, dlci 102(0x66,0x1860), broadcast
          status defined, active
Serial0/0.103 (up): point-to-point dlci, dlci 103(0x67,0x1870), broadcast
          status defined, active
Matriz#

Matriz#show ip route
Codes: C - connected, S - static, R - RIP, M - mobile, B - BGP
       D - EIGRP, EX - EIGRP external, O - OSPF, IA - OSPF inter area
       N1 - OSPF NSSA external type 1, N2 - OSPF NSSA external type 2
       E1 - OSPF external type 1, E2 - OSPF external type 2
       i - IS-IS, su - IS-IS summary, L1 - IS-IS level-1, L2 - IS-IS level-2
       ia - IS-IS inter area, * - candidate default, U - per-user static route
       o - ODR, P - periodic downloaded static route

Gateway of last resort is not set

     200.1.1.0/30 is subnetted, 2 subnets
C       200.1.1.0 is directly connected, Serial0/0.102
C       200.1.1.4 is directly connected, Serial0/0.103
C    200.10.1.0/24 is directly connected, Loopback0
     200.100.1.0/32 is subnetted, 1 subnets
O       200.100.1.1 [110/65] via 200.1.1.6, 00:06:33, Serial0/0.103
     200.134.10.0/32 is subnetted, 1 subnets
O       200.134.10.1 [110/65] via 200.1.1.2, 00:06:33, Serial0/0.102

Matriz#show frame-relay pvc

PVC Statistics for interface Serial0/0 (Frame Relay DTE)

              Active     Inactive      Deleted       Static
  Local          2            0            0            0
  Switched       0            0            0            0
  Unused         0            0            0            0

DLCI = 102, DLCI USAGE = LOCAL, PVC STATUS = ACTIVE, INTERFACE = Serial0/0.102

  input pkts 119           output pkts 130          in bytes 19887
  out bytes 21642          dropped pkts 0           in pkts dropped 0
  out pkts dropped 0                out bytes dropped 0
  in FECN pkts 0           in BECN pkts 0           out FECN pkts 0
  out BECN pkts 0          in DE pkts 0             out DE pkts 0
  out bcast pkts 110       out bcast bytes 19562
  5 minute input rate 0 bits/sec, 0 packets/sec
  5 minute output rate 0 bits/sec, 0 packets/sec
  pvc create time 00:54:05, last time pvc status changed 00:39:42

DLCI = 103, DLCI USAGE = LOCAL, PVC STATUS = ACTIVE, INTERFACE = Serial0/0.103

  input pkts 101           output pkts 120          in bytes 16648
  out bytes 20555          dropped pkts 0           in pkts dropped 0
  out pkts dropped 0                out bytes dropped 0
  in FECN pkts 0           in BECN pkts 0           out FECN pkts 0
  out BECN pkts 0          in DE pkts 0             out DE pkts 0
  out bcast pkts 105       out bcast bytes 18995
  5 minute input rate 0 bits/sec, 0 packets/sec
  5 minute output rate 0 bits/sec, 0 packets/sec
  pvc create time 00:53:14, last time pvc status changed 00:11:46
Matriz#
\end{verbatim}

\hypertarget{roteador}{%
\section{Roteador}\label{roteador}}

\begin{verbatim}
R> enable
R(config)# config terminal
R(config)# ipv6 unicast-routing 
R(config)# interface fa 0/1
R(config-if)# ipv6 enable
R(config-if))# ipv6 address 2001:dbb1:1:1::1/64
R(config-if)# no shut
R(config-if)# exit
\end{verbatim}

\hypertarget{ativar-o-rip}{%
\subsection{ativar o rip}\label{ativar-o-rip}}

\hypertarget{na-interface}{%
\subsubsection{Na interface}\label{na-interface}}

\begin{verbatim}
R(config-if)# ipv6 rip nrede enable
\end{verbatim}

\hypertarget{na-configurauxe7uxe3o-global}{%
\subsubsection{Na configuração
global}\label{na-configurauxe7uxe3o-global}}

Onde \emph{nrede} é uma palavra chave

\begin{verbatim}
R (config) # ipv6 router rip nrede
\end{verbatim}

\hypertarget{nat}{%
\section{NAT}\label{nat}}

\begin{verbatim}
Router(config)# int fa 0/1
Router(config)# ip nat outside
Router(config)# exit 
Router(config)# interface fa 0/0
Router(config)# ip address ..
Router(config)# ip nat inside 
Router(config)# access-list 1 permit 192.168.1.0 0.0.0.255
Router(config)# access-list 1 permit 200.1.1.0 0.0.0.3
Router(config)# ip nat inside source list 1
Router(config)# interface fa 0/1
Router(config)# overload
\end{verbatim}

\hypertarget{qos}{%
\section{QoS}\label{qos}}

\hypertarget{configurauxe7uxe3o-de-uma-rede-voip}{%
\subsection{Configuração de uma rede
VoIP}\label{configurauxe7uxe3o-de-uma-rede-voip}}

Vamos configurar uma rede com a seguinte topologia:

\begin{figure}
\centering
\includegraphics{imagens/voip/voip-configuracao.png}
\caption{Topologia}
\end{figure}

\hypertarget{configurauxe7uxe3o-do-switch}{%
\subsubsection{Configuração do
Switch}\label{configurauxe7uxe3o-do-switch}}

\begin{verbatim}
Switch(config)#interface range fa0/1 – 5
Switch(config-if-range)#switchport mode access
Switch(config-if-range)#switchport voice vlan 1
\end{verbatim}

\hypertarget{configurauxe7uxe3o-do-roteador-2811-com-o-cme}{%
\subsubsection{Configuração do roteador 2811 com o
CME:}\label{configurauxe7uxe3o-do-roteador-2811-com-o-cme}}

\begin{verbatim}
Router(config)#int fa 0/0
Router(config-if)#ip add 192.168.10.1 255.255.255.0
Router(config-if)#no shutdown
Router(config-if)#exit
Router(config)#ip dhcp pool voicelab
Router(dhcp-config)#network 192.168.10.0 255.255.255.0
Router(dhcp-config)#default-router 192.168.10.1
Router(dhcp-config)#option 150 ip 192.168.10.1
Router(dhcp-config)#exit
Router(config)#telephony-service
Router(config-telephony)#max-dn 5
Router(config-telephony)#max-ephones 5
Router(config-telephony)#ip source-address 192.168.10.1 port 2000
Router(config-telephony)#auto assign 1 to 5
Router(config-telephony)#exit
Router(config)#ephone-dn 1
Router(config-ephone-dn)#number 54001
Router(config-ephone-dn)#exit
Router(config)#ephone-dn 2
Router(config-ephone-dn)#number 54002
Router(config-ephone-dn)#
Router(config)#ephone-dn 3
Router(config-ephone-dn)#number 11111
\end{verbatim}

\hypertarget{configurauxe7uxe3o-da-qualidade-de-serviuxe7o-qos-no-switch-2960}{%
\subsubsection{Configuração da Qualidade de Serviço (QoS) no Switch
2960}\label{configurauxe7uxe3o-da-qualidade-de-serviuxe7o-qos-no-switch-2960}}

\begin{verbatim}
Switch#configure terminal
Switch(config)#mls qos
Switch(config)#interface range fastEthernet 0/1-5
Switch(config-if-range)#mls qos
Switch(config-if-range)#mls qos cos 5
Switch(config-if-range)#mls qos trust cos
Switch(config-if-range)#exit
Switch(config)#interface range fastEthernet 0/10-22
Switch(config-if-range)#mls qos
Switch(config-if-range)#mls qos cos 0
Switch(config-if-range)#mls qos trust cos
Switch(config-if-range)#exit
Switch(config)#end
Switch#wr
\end{verbatim}

\hypertarget{policing}{%
\section{Policing}\label{policing}}

\hypertarget{topologia}{%
\subsection{Topologia}\label{topologia}}

\begin{figure}
\centering
\includegraphics{imagens/qos/policing.png}
\caption{Topologia}
\end{figure}

\hypertarget{buxe1sico}{%
\subsection{Básico}\label{buxe1sico}}

\begin{verbatim}
Rot1(config)#interface fa 0/1
Rot1(config-if)#ip address dhcp
Rot1(config-if)#description Link para UTFPR
Rot1(config-if)#ip nat outside
Rot1(config-if)#no shutdown
Rot1(config-if)#exit
Rot1(config)#interface fa 0/0
Rot1(config-if)#no shutdown
Rot1(config-if)#exit
Rot1(config)#interface fa 0/0.10
Rot1(config-if)#encapsulation dot1q 10
Rot1(config-if)#ip address 192.168.100.1 255.255.255.0
Rot1(config-if)#ip nat inside
Rot1(config-if)#exit
Rot1(config)#interface fa 0/0.20
Rot1(config-if)#encapsulation dot1q 20
Rot1(config-if)#ip address 192.168.200.1 255.255.255.0
Rot1(config-if)#ip nat inside
Rot1(config-if)#exit
Rot1(config)#ip route 0.0.0.0 0.0.0.0 10.15.2.254
Rot1(config)#access-list 1 permit 192.168.0.0 0.0.255.255
Rot1(config)#ip nat inside source list 1 interface fa 0/1 overload
\end{verbatim}

\hypertarget{configurauxe7uxe3o-policing}{%
\subsection{Configuração Policing}\label{configurauxe7uxe3o-policing}}

\begin{verbatim}
Rot1(config)#ip access-list extended HostA
Rot1(config-ext-nacl)#permit ip any host 192.168.100.2
Rot1(config-ext-nacl)#permit ip host 192.168.100.2 any
Rot1(config-ext-nacl)#exit
Rot1(config)#class-map match-all HA
Rot1(config-cmap)#match access-group name HostA
Rot1(config-cmap)#exit
Rot1(config)#policy-map QoS1
Rot1(config-pmap)#class HA
* Rot1(config-pmap-c)#police rate 64000 bps
Rot1(config-pmap-c-police)#end
Rot1#config terminal
Rot1(config)#interface fa 0/0.10
Rot1(config-if)#service-policy output QoS1
\end{verbatim}

\hypertarget{configurauxe7uxe3o-shapping}{%
\subsubsection{Configuração
Shapping}\label{configurauxe7uxe3o-shapping}}

\begin{itemize}
\tightlist
\item
  Ao invés de police rate \ldots{} Na linha
\end{itemize}

\begin{verbatim}
Rot1(config-pmap-c)#police rate 64000 bps
\end{verbatim}

\begin{itemize}
\tightlist
\item
  Colocar:
\end{itemize}

\begin{verbatim}
Rot1(config-pmap-c)#shape average 128000
\end{verbatim}

\hypertarget{continuauxe7uxe3o}{%
\subsection{Continuação\ldots{}}\label{continuauxe7uxe3o}}

\begin{verbatim}
Rot1(config)#ip access-list extended HostB
Rot1(config-ext-nacl)#permit ip any host 192.168.200.2
Rot1(config-ext-nacl)#permit ip host 192.168.200.2 any
Rot1(config-ext-nacl)#exit
Rot1(config)#class-map match-all HB
Rot1(config-cmap)#match access-group name HostB
Rot1(config-cmap)#exit
Rot1(config)#policy-map QoS2
Rot1(config-pmap)#class HB
Rot1(config-pmap-c)#police rate 2000000 bps
Rot1(config-pmap-c-police)#end
Rot1#config terminal
Rot1(config)#interface fa 0/0.20
Rot1(config-if)#service-policy output QoS2
\end{verbatim}

\hypertarget{eigrp-1}{%
\section{EIGRP 1}\label{eigrp-1}}

\begin{verbatim}
R(config)#router eigrp 1
R(config-router)#network 192.168.1.0
R(config-router)#network 200.1.1.0
R(config-router)#end
R#wr
\end{verbatim}

\hypertarget{ospf}{%
\section{OSPF}\label{ospf}}

\emph{Atenção: a máscara é invertida, i.e., wildmask}

ex: /24 Ao invés de ser 255.255.255.0 é 0.0.0.255

\begin{verbatim}
R(config)# router ospf 1
R(config-router)# network 192.168.1.0 0.0.0.255 area 0
R(config-router)# end
\end{verbatim}

\hypertarget{seguranuxe7a-de-porta}{%
\section{Segurança de porta}\label{seguranuxe7a-de-porta}}

\hypertarget{violauxe7uxf5es}{%
\subsection{Violações:}\label{violauxe7uxf5es}}

\begin{verbatim}
* Protect

* Restrict

* Shutdown
\end{verbatim}

\hypertarget{observauxe7uxf5es}{%
\subsection{Observações:}\label{observauxe7uxf5es}}

\begin{verbatim}
* Somente modo acesso

* Existe estático e dinâmico
\end{verbatim}

\hypertarget{estuxe1tico}{%
\subsection{Estático:}\label{estuxe1tico}}

\begin{verbatim}
switchport port-security mac-address AA:AA:AA:AA:00:00:00:01
switchport maximum 1
switchport violation shutdown #desliga a interface se houver uma violação
\end{verbatim}

\hypertarget{dinuxe2mico}{%
\subsection{Dinâmico:}\label{dinuxe2mico}}

\begin{verbatim}
Apenas não específicar o mac
\end{verbatim}

\begin{verbatim}
switchport maximum 100 
switchport violation shutdown #desliga a interface se houver uma violação
\end{verbatim}

\hypertarget{dhcp---snooping}{%
\section{DHCP - snooping}\label{dhcp---snooping}}

\begin{verbatim}
ip dhcp snooping
ip dhcp snooping vlan 10
inteface range f0/1-10
ip dhcp snooping limit rate 5
interface g0/1
ip dhcp snooping trust
end
show ip dhcp snooping
\end{verbatim}

\hypertarget{ssh}{%
\section{SSH}\label{ssh}}

\begin{verbatim}
S# vlan 30 
S# ip domain-name X 
S# username fabiano privilege 15 password CISCO
S#line vty 0 4
#transport input telnet/ssh/all 

#enable secret UTFPR (pede a senha pra entrar no modo privilegiado) 
#hostname SwitchX 
#ip domain-name www.utfpr.edu.br 
#username NOME priv 15 password UTFPR 
#crypto key generate rsa 
#line vty 0 4 (5 conexões simultaneas) 
#transport input ssh 
#login local 
\end{verbatim}

(para acesso remoto das vlans acesse vlan.md)

\hypertarget{stp}{%
\section{STP}\label{stp}}

\hypertarget{no-switch-que-seruxe1-o-root}{%
\subsection{No switch que será o
root}\label{no-switch-que-seruxe1-o-root}}

\begin{verbatim}
Switch>enable
Switch#config terminal
Switch(config)#spanning-tree vlan 1 root primary
Switch(config)#end
Switch#
\end{verbatim}

\hypertarget{nos-switches-que-seruxe3o-secunduxe1rios}{%
\subsection{Nos switches que serão
secundários}\label{nos-switches-que-seruxe3o-secunduxe1rios}}

\begin{verbatim}
Switch>enable
Switch#config terminal
Switch(config)#spanning-tree vlan 1 root secondary
Switch(config)#end
Switch#
Switch#
\end{verbatim}

\hypertarget{para-configurar-o-switch-na-topologia-desativando-o-protocolo-spanning-tree-stp.-tem-somente-a-vlan-1-padruxe3o-configurada.}{%
\subsection{Para configurar o switch na topologia desativando o
protocolo Spanning Tree (STP). Tem somente a Vlan 1 (padrão)
configurada.}\label{para-configurar-o-switch-na-topologia-desativando-o-protocolo-spanning-tree-stp.-tem-somente-a-vlan-1-padruxe3o-configurada.}}

\begin{verbatim}
Switch>enable
Switch#config terminal
Enter configuration commands, one per line. End with CNTL/Z.
Switch(config)#no spanning-tree vlan 1
Switch(config)#end
Switch#
wr
Building configuration...
[OK]
Switch#
\end{verbatim}

\hypertarget{tfpt}{%
\section{TFPT}\label{tfpt}}

\begin{itemize}
\tightlist
\item
  Arquivo de configuração do TFTP:
\end{itemize}

\texttt{/etc/default/tftpd-hpa}

\begin{itemize}
\tightlist
\item
  Pasta padrão do servidor tftp:
\end{itemize}

\texttt{/home/tftp/}

\begin{itemize}
\tightlist
\item
  Iniciar, reinicializar ou desligar o deamon do TFTP:
\end{itemize}

\begin{verbatim}
/etc/init.d/tftpd-hpa {start|restart|stop|force-reload|status}
\end{verbatim}

\begin{itemize}
\tightlist
\item
  \emph{Exemplo}: copiar o arquivo running-config do sistema para o
  servidor
\end{itemize}

\begin{verbatim}
copy system:running-config tftp
\end{verbatim}

\hypertarget{tunelamento}{%
\section{TUNELAMENTO}\label{tunelamento}}

\hypertarget{tuxfanel-6in4}{%
\subsection{Túnel 6in4}\label{tuxfanel-6in4}}

\begin{verbatim}
router> enable
router# config t
router(config)# interface tunnel 0
router(config-if)# ipv6 address fd00:cafe::0/127
router(config-if)# tunnel source serial x/x/x
router(config-if)# tunnel destination 203.0.113.6
router(config-if)# tunnel mode ipv6ip
router(config-if)# ipv6 route 2001:db8:cafe:2::/64 fd00:cafe::1
\end{verbatim}

\hypertarget{uxfatil}{%
\section{Útil}\label{uxfatil}}

Comandos que podem ser úteis no gerenciamento e configuração das redes

\textbf{Range de intefaces}

\begin{verbatim}
# interface range fa 0/0-10
\end{verbatim}

\hypertarget{duxe1-pra-separar-por-vuxedrgulas}{%
\subsection{Dá pra separar por
vírgulas}\label{duxe1-pra-separar-por-vuxedrgulas}}

\begin{verbatim}
# interface range fa 0/1, fa 0/3, fa0/5
\end{verbatim}

\textbf{Mostra configuração geral}

\begin{verbatim}
# show running-config
\end{verbatim}

\textbf{Mostra as vlans e suas intefaces designadas}

\begin{verbatim}
# show vlan brief
\end{verbatim}

\begin{verbatim}
# show vlan
\end{verbatim}

\textbf{Mostra a tabela MAC \textbar{} INTERFACE do switch}

\begin{verbatim}
show mac-addres table
\end{verbatim}

\textbf{Hostname}

\begin{verbatim}
#hostname zoera
\end{verbatim}

\textbf{DEBUG}

\begin{verbatim}
debug <o que você quer debugar>
\end{verbatim}

\emph{exemplos}

\begin{verbatim}
debug arp
debug dhcp
debug port-security
debug all
\end{verbatim}

OBS.: \texttt{debug\ all} não é recomendado

\textbf{Sair do debug}

\begin{verbatim}
undebug <o que você quer desbugar>
\end{verbatim}

\begin{verbatim}
undebug arp
undebug dhcp
undebug ppp
undebug all
\end{verbatim}

\hypertarget{gravauxe7uxf5esexclusuxf5es-de-configurauxe7uxe3o}{%
\subsection{Gravações/exclusões de
configuração}\label{gravauxe7uxf5esexclusuxf5es-de-configurauxe7uxe3o}}

\texttt{Ram} armazena running-config

\texttt{nVran} armazena startup-config

\texttt{Flash} armazena SO e vlan.dat (conteúdo das vlans)

\hypertarget{gravar-ram---vram}{%
\subsubsection{Gravar Ram -\textgreater{}
vRam}\label{gravar-ram---vram}}

\begin{verbatim}
#copy running-config startup-config 
\end{verbatim}

OR

\begin{verbatim}
#wr 
\end{verbatim}

\hypertarget{apagar-tudo}{%
\subsubsection{Apagar tudo:}\label{apagar-tudo}}

\begin{verbatim}
#erase startup-config 
#delete flash:vlan.dat 
#reload 
\end{verbatim}

\hypertarget{recuperauxe7uxe3o-de-dispositivos}{%
\section{Recuperação de
dispositivos}\label{recuperauxe7uxe3o-de-dispositivos}}

\hypertarget{recuperar-senhas}{%
\subsection{Recuperar senhas}\label{recuperar-senhas}}

\hypertarget{no-roteador}{%
\subsubsection{No roteador}\label{no-roteador}}

\begin{enumerate}
\def\labelenumi{\arabic{enumi}.}
\item
  Fazer a conexão com o equipamento utilizando o cabo serial e o kermit;
\item
  Reinicializar fisicamente o roteador;
\item
  Acessar o modo Rommon no roteador: Após 5 segundos, pressionar
  simultaneamente as teclas Ctrl-\l ou Ctrl-\b. Este procedimento
  interrompe a sequência normal do boot e inicia o Rom Monitor.
\end{enumerate}

\begin{verbatim}
rommon >
\end{verbatim}

\begin{enumerate}
\def\labelenumi{\arabic{enumi}.}
\setcounter{enumi}{3}
\tightlist
\item
  Alterar o registro de configuração para o valor 0x2142. Com isto, na
  reinicialização do roteador o equipamento não vai carregar a
  configuração da NVRAM. Com isto as configurações salvas não serão
  carregadas e não haverá senha para entrar no roteador.
\end{enumerate}

\begin{verbatim}
rommon> confreg 0x2142
rommon> reset
\end{verbatim}

\begin{enumerate}
\def\labelenumi{\arabic{enumi}.}
\setcounter{enumi}{4}
\tightlist
\item
  Apenas digitar \emph{enable}, não precisa digitar a senha
\end{enumerate}

\begin{verbatim}
router> enable
router# copy startup-config running-config
\end{verbatim}

\begin{itemize}
\tightlist
\item
  A partir deste ponto toda a configuração da NVRAM estará na RAM e
  poderá ser alterada, inclusive a senha.
\end{itemize}

\hypertarget{no-switch}{%
\subsubsection{No switch}\label{no-switch}}

\begin{enumerate}
\def\labelenumi{\arabic{enumi}.}
\item
  Desligar o switch da tomada e segurar o botão switch old.
\item
  Nesse modo, existem 3 comandos:
\end{enumerate}

\begin{verbatim}
flash_init

load_helper

boot
\end{verbatim}

\begin{itemize}
\tightlist
\item
  Fazer os comandos nessa ordem, flash\_init, load\_helper, então
  renomear o arquivo config.text e, então, executar o comando boot:
\end{itemize}

\begin{verbatim}
flash_init
load_helper
rename flash:config.text flash:config.old
boot
\end{verbatim}

\begin{itemize}
\item
  Então você será logado ao sistema normal, como se tivesse comprado
  agora o dispositivo.
\item
  Renomear, por fim, o arquivo config.old para config.text, depois
  copiar config.text pra running-config:
\end{itemize}

\begin{verbatim}
copy flash:config.old flash:config.text
copy flash:config.text system:running-config
\end{verbatim}

\hypertarget{ios-apagado}{%
\subsection{IOS apagado}\label{ios-apagado}}

\begin{itemize}
\item
  Setar as configurações de rede do roteador, iniciar servidor tftp,
  setar o nome do .bin em um servidor tftp, setar ip de servidor tftp e
  baixar o arquivo pela rede.
\item
  \textbf{EXEMPLO}
\end{itemize}

\begin{verbatim}
rommon > IP_SUBNET_MASK=255.255.255.0
rommon > DEFAULT_GATEWAY=171.68.170.3
rommon > TFTP_FILE=c2600-is-mz.113.2.0
rommon > tftpdn1d
\end{verbatim}

\hypertarget{vlans}{%
\section{VLANS}\label{vlans}}

\hypertarget{criar-vlans}{%
\subsection{Criar vlans}\label{criar-vlans}}

\begin{verbatim}
S(config)# vlan <1-1005>
S(config-vlan)# name avelã
\end{verbatim}

\hypertarget{access-mode}{%
\subsection{Access Mode}\label{access-mode}}

\begin{verbatim}
S(config-if)# switchport mode access
S(config-if)# switchport access vlan <1-1005>
\end{verbatim}

\hypertarget{trunk-mode}{%
\subsection{Trunk Mode}\label{trunk-mode}}

\begin{verbatim}
S(config-if)# switchport mode trunk
S(config-if)# switchport trunk allowed vlan <1-1005>
\end{verbatim}

\begin{verbatim}
Há também os comandos add, all, onde all adiciona uma nova vlan na lista atual, e all adiciona todas.
\end{verbatim}

\hypertarget{roteamento-de-vlans}{%
\subsection{Roteamento de vlans}\label{roteamento-de-vlans}}

\hypertarget{sem-subintefaces}{%
\subsection{Sem subintefaces}\label{sem-subintefaces}}

\begin{verbatim}
* 1 interface pra cada vlan

* Switch -> roteador
- Mode *acess* e vlan que irá passar pela inteface 

* Roteador -> Switch
- Seta o ip da interface pra cada vlan
\end{verbatim}

\hypertarget{com-subintefaces}{%
\subsection{Com subintefaces}\label{com-subintefaces}}

\begin{itemize}
\item
  Uma interface para várias vlans
\item
  Switch -\textgreater{} roteador

  \begin{itemize}
  \tightlist
  \item
    Tronco e autorizar todas as vlans daquela interface/
  \end{itemize}
\item
  Roteador -\textgreater{} switch

  \begin{itemize}
  \item
    Subinterfaces;
  \item
    Só dá um ``no shut'' para subir a interface
  \item
    Encapsulation dot1q nas subinterfaces
  \item
    Gateway default de todas as vlans
  \end{itemize}
\end{itemize}

\hypertarget{comandos-1}{%
\subsubsection{Comandos}\label{comandos-1}}

\begin{verbatim}
router>enable
router#config terminal
router(config)#interface fa 0/0
router(config-if)#no shutdown
router(config)#interface fa 0/0.10
router(config-if)#encapsulation dot1q 10
router(config-if)#ip address 10.0.0.1 255.0.0.0
router(config-if)#exit
router(config)#interface fa 0/0.20
router(config-if)#encapsulation dot1q 20
router(config-if)#ip address 172.17.0.1 255.255.0.0
router(config-if)#exit
router(config)#interface fa 0/0.30
router(config-if)#encapsulation dot1q 30
router(config-if)#ip address 192.168.1.1 255.255.255.0
\end{verbatim}

\begin{verbatim}
switch>enable
switch#config terminal
switch(config)#vlan 10
switch(config-vlan)#name Funcionarios
switch(config-vlan)#vlan 20
switch(config-vlan)#name Convidados
switch(config-vlan)#vlan 30
switch(config-if)#name Gerencia
switch(config-vlan)#exit
switch(config)#interface fa 0/1
switch(config-if)#switchport mode access
switch(config-if)#switchport mode access
switch(config-if)#switchport access vlan 10
switch(config-if)#exit
switch(config)#interface fa 0/24
switch(config-if)#switchtport mode trunk
switch(config-if)#switchport trunk vlan 10,20,30
\end{verbatim}

\hypertarget{ip-da-vlan-para-acesso-remoto}{%
\section{IP da vlan para acesso
remoto}\label{ip-da-vlan-para-acesso-remoto}}

\begin{verbatim}
#interface vlan 20 
#ip address 200.1.1.200 255.255.255.0 
#no shut 
\end{verbatim}

\hypertarget{protocolos-da-camada-2}{%
\section{Protocolos da camada 2}\label{protocolos-da-camada-2}}

\hypertarget{analogia-da-diferenuxe7a-do-pap-e-chap}{%
\subsection{Analogia da diferença do PAP e
CHAP}\label{analogia-da-diferenuxe7a-do-pap-e-chap}}

\texttt{Pense\ como\ a\ diferença\ do\ ssh\ e\ do\ telnet,\ analogamente}

\hypertarget{ppp-com-pap}{%
\subsection{PPP com PAP}\label{ppp-com-pap}}

PPP é um protocolo aberto

\begin{verbatim}
R1 (config)#    username R2
        password utfpr
R1 (config)# inter se 0/0/0
R1 (config-if)# encapsulation ppp
R1 (config-if)# ppp pap sent-username R1 password utfpr
\end{verbatim}

\hypertarget{ppp-com-chap}{%
\subsection{PPP com CHAP}\label{ppp-com-chap}}

\begin{verbatim}
R2 (config)# interface serial 0/1/0
R2 (config)# encapsulation ppp
R2 (config)# ppp authentication chap
R2 (config)# exit
R2 (config)# username R2 password utfpr
R2 (config)# username R3 password utfpr
\end{verbatim}

\hypertarget{hdlc}{%
\subsection{HDLC}\label{hdlc}}

\begin{verbatim}
R3 (config)# interface serial 0/0/0
R3 (config-if)# encapsulation hdlc
\end{verbatim}

\end{document}
